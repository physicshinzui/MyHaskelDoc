\section{Terminology}
Technical terms often used in Haskell

\paragraph{Top level (global) variable}
Define before \lstinline{main}

\begin{lstlisting}
x=1
main = do print (x) 
\end{lstlisting}

\paragraph{Local variable}
Define after \lstinline{where}
\begin{lstlisting}
main = do 
    print (x)
    where 
        x = 1
\end{lstlisting}

Or define in \lstinline{let}
\begin{lstlisting}
main = do
    let x = 1 
    print (x)
\end{lstlisting}

\paragraph{Prefix, Infix Function} 
Prefix function is the one which takes arguments/variables after the function name, 
e.g., \lstinline{f x = x*2}. 

Infix function is the one which takes those before and after the function name, 
e.g.,  \lstinline{4 `elem` [2,4,6]} and \lstinline{91 `div` 10}, which can also written by \lstinline{elem 4 [2,4,6]}  and \lstinline{div 91 10}, 
but said not not to be sweet in Haskell [cite XXX] because it is ambiguous how the function works from which to which. 
By the way, div returns integral division of divided value. 

The basic operators  \lstinline{+, -, *, /} are also infix functions.

\paragraph{Syntax sugar}
Syntax sugar means an easier way to express codes. Let’s say a list in Haskell is expressed as  
\lstinline{[1,2,3,4]} which is a syntax sugar for \lstinline{1:2:3:4:[]}. Also, function expression does have also syntax sugar like \lstinline{f x y = x*y} as sugar for 
\begin{lstlisting}
f = \x y -> x*y
f = \x -> \y -> x*y
\end{lstlisting}

Syntax sugar is explained in  \href{https://en.wikibooks.org/wiki/Haskell/Syntactic_sugar#Function_Bindings}{'here'}
\paragraph{Currying [fixme]}

Currying is explained here:
\href{https://wiki.haskell.org/Currying}{Currying}

\paragraph{Referential transparency [fixme]}
