\begin{lstlisting}
--Task---
--multiple of three -> Fizz
--multiple of five -> Buzz
--multiple of both three and five -> FizzBuzz

--returnFizz :: [Int] -> [Int]
--returnFizz set = map (\x -> if x `mod` 3 == 0 then "Fizz" else x ) set
--note that this does not work because the returned list has potentially not only numbers but also String.
--List cannot have different type in it.

filterFizzBuzz :: Int -> String
filterFizzBuzz x =
    if       x `mod` 15 == 0 then "FizzBuzz"
    else if  x `mod` 5  == 0 then "Buzz"
    else if  x `mod` 3  == 0 then "Fizz"
    else show x

returnFizzBuzz :: [Int] -> [String]
returnFizzBuzz set = map (filterFizzBuzz) set

fizzBuzz :: Int -> String
fizzBuzz x
    | x `mod` 15 == 0 = "FizzBuzz"
    | x `mod` 5  == 0 = "Buzz"
    | x `mod` 3  == 0 = "Fizz"
    | otherwise = show x

fizzBuzz' :: [Int] -> [String]
fizzBuzz' intList = map (fizzBuzz) intList

main = do
    let x = [1..100] in print $ fizzBuzz' x
 \end{lstlisting}
 
\subsection{Random number}
We can use \lstinline{System.Random} library \footnote{This library is not good for cryptograph for which we should use \lstinline{Crypto.Random}. Also, someone said that it has some performance issues.}  to generate random numbers.

\begin{itemize}
\item \lstinline{mkStdGen}: returns a generator that was set when opened
\item \lstinline{getStdGen}: same?
\item \lstinline{newStdGen}: returns a new generator.
\item \lstinline{randomR}: Take two arguments \lstinline{(min,max)} and a generator \lstinline{g} from \lstinline{g <- mkStdGen or getStdGen}. It returns a random number.
\item \lstinline{randomRs}: the same arguments, but returns a list of random numbers.
\end{itemize}

\paragraph{Example: Print five random numbers}

\begin{lstlisting}
import System.Random
main =  do x <- newStdGen
	        print $ take 5 $ randomRs (0,10) x 
\end{lstlisting}
